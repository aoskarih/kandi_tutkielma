% STEP 1: Choose oneside or twoside
\documentclass[finnish,twoside,openright]{HYgradu}

%\usepackage[utf8]{inputenc} % For UTF8 support. Use UTF8 when saving your file.
\usepackage{lmodern} % Font package
\usepackage{textcomp} % Package for special symbols
\usepackage[pdftex]{color, graphicx} % For pdf output and jpg/png graphics
\usepackage[pdftex, plainpages=false]{hyperref} % For hyperlinks and pdf metadata
\usepackage{fancyhdr} % For nicer page headers
\usepackage{tikz} % For making vector graphics (hard to learn but powerful)
%\usepackage{wrapfig} % For nice text-wrapping figures (use at own discretion)
\usepackage{amsmath, amssymb} % For better math
%\usepackage[square]{natbib} % For bibliography
\usepackage[footnotesize,bf]{caption} % For more control over figure captions
\usepackage{blindtext}
\usepackage{titlesec}
\usepackage[titletoc]{appendix}

\onehalfspacing %line spacing
%\singlespacing
%\doublespacing

%\fussy 
\sloppy % sloppy and fussy commands can be used to avoid overlong text lines

% STEP 2:
% Set up all the information for the title page and the abstract form.
% Replace parameters with your information.
\title{Symplektiset integrointimenetelm�t}
\author{Arttu Hyv�nen}
\date{\today}
%\level{Bachelor's thesis}
\level{Kandidaatintutkielma}
\subject{Teoreettinen fysiikka}
%\subject{Your Field}
\faculty{Matemaattis-luonnontieteellinen tiedekunta}
%\faculty{Faculty of Whatever}
\programme{Fysikaalisten tieteiden kandiohjelma}
\department{Fysiikan laitos}
%\department{Department of Something}
\address{PL 64 (Gustaf H�llstr�min katu 2a)\\00014 Helsingin yliopisto}
\prof{Pauli Pihajoki}
\censors{arvostelija Testi}{arvostelija Arvostelija}{}
\keywords{\LaTeX}
\depositeplace{}
\additionalinformation{}
\classification{}

% if you want to quote someone special. You can comment this line and there will be nothing on the document.
%\quoting{Bachelor's degrees make pretty good placemats if you get them laminated.}{Jeph Jacques} 


% OPTIONAL STEP: Set up properties and metadata for the pdf file that pdfLaTeX makes.
% But you don't really need to do this unless you want to.
\hypersetup{
    bookmarks=true,         % show bookmarks bar first?
    unicode=true,           % to show non-Latin characters in Acrobat’s bookmarks
    pdftoolbar=true,        % show Acrobat’s toolbar?
    pdfmenubar=true,        % show Acrobat’s menu?
    pdffitwindow=false,     % window fit to page when opened
    pdfstartview={FitH},    % fits the width of the page to the window
    pdftitle={},            % title
    pdfauthor={},           % author
    pdfsubject={},          % subject of the document
    pdfcreator={},          % creator of the document
    pdfproducer={pdfLaTeX}, % producer of the document
    pdfkeywords={something} {something else}, % list of keywords for
    pdfnewwindow=true,      % links in new window
    colorlinks=true,        % false: boxed links; true: colored links
    linkcolor=black,        % color of internal links
    citecolor=black,        % color of links to bibliography
    filecolor=magenta,      % color of file links
    urlcolor=cyan           % color of external links
}

\begin{document}

% Generate title page.
\maketitle

% STEP 3:
% Write your abstract (of course you really do this last).
% You can make several abstract pages (if you want it in different languages),
% but you should also then redefine some of the above parameters in the proper
% language as well, in between the abstract definitions.

\begin{abstract}
Kirjoita tiivistelm��n lyhyt, enint��n 250 sanan yhteenveto ty�st�si: mit� olet tutkinut, millaisia menetelmi� olet k�ytt�nyt, millaisia tuloksia sait ja millaisia johtop��t�ksi� niiden perusteella voi tehd�.
\end{abstract}

% Place ToC
\mytableofcontents

%\mynomenclature

% -----------------------------------------------------------------------------------
% STEP 4: Write the thesis.
% Your actual text starts here. You shouldn't mess with the code above the line except
% to change the parameters. Removing the abstract and ToC commands will mess up stuff.
\chapter{Johdanto}


\chapter{Liikeyht�l�t ja Hamilton}



\section{Systeemi}

Systeemi(t) joita ty�ss� tarkastellaan ja analyyttiset ratkaisut.

\section{Hamilton}

Yht�l�t, joita ruvetaan ratkaisemaan.


\chapter{Runge-Kutta}

\section{Johto}

Runge-Kuttan johto.

\section{Toteutus}

Menetelm�n toteutus.

\chapter{Loikkakeino}

\section{Johto}

Loikkakeinon johto.

\section{Toteutus}

Menetelm�n toteutus.

\chapter{Vertailu}

Vertaillaan Runge-Kuttaa, loikkakeinoa ja analyyttist� ratkaisua.


\chapter{P��telm�t}

Mit� johtop��t�ksi� voidaan tehd� tuloksista ja vertailusta.


\chapter{Liitteet}
Liitteiss� voi esitell� esimerkiksi ty�ss� k�ytettyj� tietokonekoodeja:


% STEP 5:
% Uncomment the following lines and set your .bib file and desired bibliography style
% to make a bibliography with BibTeX.
% Alternatively you can use the thebibliography environment if you want to add all
% references by hand.

\cleardoublepage %fixes the position of bibliography in bookmarks
\phantomsection

\addcontentsline{toc}{chapter}{\bibname} % This lines adds the bibliography to the ToC
%\bibliographystyle{unsrt} % numbering 
\bibliographystyle{apalike} % name, year
\bibliography{bibliography.bib}

\end{document}
